\documentclass[a4paper, twocolumn, 10pt]{article}

%Paquetes idioma
\usepackage[utf8]{inputenc}
\usepackage[spanish]{babel}

%Paquetes matemáticas
\usepackage{amsmath}
\usepackage{amsfonts}
\usepackage{amssymb}

%Paquetes variados y organización
\usepackage{graphicx}
\usepackage[left=2cm,right=2cm,top=2cm,bottom=2cm]{geometry}
\usepackage{multicol}
\usepackage{float}
\usepackage{multirow}
\usepackage{subfigure}
\usepackage{cite}
\usepackage{stfloats}
\usepackage{array}
%Paquetes matemáticas
\usepackage{amsmath}
\usepackage{amsfonts}
\usepackage{amssymb}
\usepackage{listings}
\usepackage{color}
\usepackage{graphicx} % figuras
\usepackage{subfigure} % subfiguras
\usepackage{paralist} % Used for the compactitem environment which makes bullet points with less space between them
\usepackage[none]{hyphenat}
\usepackage{afterpage}


%Paquetes variados y organización
\usepackage{graphicx}
\usepackage[left=2cm,right=2cm,top=2cm,bottom=2cm]{geometry}
\usepackage{multicol}
\usepackage{float}
\usepackage{multirow}
\usepackage{subfigure}
\usepackage{cite}
\usepackage{stfloats}
\usepackage{array}

\usepackage{fancyhdr}
\usepackage{fontspec}
\pagestyle{fancy} % seleccionamos un estilo
\usepackage{subfig}

\usepackage[maxbibnames=99, sorting=none, backend=bibtex]{biblatex}
\addbibresource{bibio.bib}

\usepackage{fancyhdr}
\usepackage{graphicx}


\rhead{\begin{picture}(0,0) \put(-57,0){\includegraphics[width=20mm]{./uis}} \end{picture}}
\chead{ propuesta \vspace{0.1cm}}
\renewcommand{\headrulewidth}{0.5pt}



%Datos del informe
\title{\textbf{ Propuesta}} 
\author{ Angie Milena Sanchez$^1$\\ \textit{Universidad Industrial de Santander$^1$}}

\begin{document}
\sloppy 
\twocolumn[
\begin{@twocolumnfalse}
\maketitle
\begin{abstract}

%\noindent{La partícula con carga eléctrica en un campo magnético describe un movimiento circular uniforme, gracias a que la fuerza magnética es  perpendicular a la velocidad de dicho cuerpo.  Para obtener la carga especifica del electrón se realizó la toma de datos de corriente para cada valor de potencial, además de la calibración de campo magnético, fue necesario hallar la pendiente de cada gráfico y así poder hacer uso de la ecuación de la carga especifica del electrón, para la cual se obtuvo un valor experimental de 5.0133$\times10^{11}$ [c/kg]}
\noindent Este trabajo consistió 

\begin{center}\rule{0.9\textwidth}{0.1mm} \end{center}
\end{abstract}
\end{@twocolumnfalse}
]
%\section*{Resumen}

\section*{Introducción}

\noindent Los 

\section*{Objetivos}

\begin{itemize}
    \item Mostrar el fundamento teórico de las metalentes.
    \item Explicar el fenómeno de la iridiscencia en diferentes insectos a partir de las características que presentan los metamateriales
\end{itemize}





\printbibliography

\end{document}
